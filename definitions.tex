\makeatletter\long\def\equals#1#2{\pdf@strcmp{#1}{#2}}\makeatother
%
\newcommand{\translate}[2]{%
  \switch%
    \case{\equals{\languagename}{english}}#1%
    \case{\equals{\languagename}{ngerman}}#2%
  \endswitch
}
% 
\newcommand{\onlyenglish}[1]{\ifcase\equals{\languagename}{english}#1\fi}
\newcommand{\onlygerman}[1]{\ifcase\equals{\languagename}{ngerman}#1\fi}
%
\newcommand{\doctitle}{\translate{Foundations of Computer Science}{Grundlagen der Informatik}}
\newcommand{\docauthor}{Lukas Prokop}
%
\title{\doctitle}
\date{\today}
\author{\docauthor}
%
\hypersetup{
  pdftitle={\doctitle},
  pdfsubject={\doctitle{} Skriptum},
  pdfauthor={\docauthor},
  pdfkeywords={\translate{Foundations, computer science, Turing, Boole, logic, complexity theory}{Grundlagen, Informatik, Turing, Boole, Logik, Komplexitätstheorie}}
}
%
\onlygerman{\usepackage{lmodern}}
%
% TODO: prefer acronyms?
% TODO: hashmap with key=CS and value gets automatically generated (first occurence expr in parenthesis added, english/german)
%
%\switch[\pdfstrcmp{\languagename}]%
\switch%
\case{\equals{\languagename}{ngerman}}%
%\case{{german}}%
  \newcommand{\studycs}{Informatik [F 033 521]}%
  \newcommand{\studysdbm}{Softwareentwicklung und Wirtschaft [F 033 524]}%
  \newcommand{\courselfocs}{,,Grundlagen der Informatik`` (VO, 716.232)}%
  \newcommand{\coursefocs}{,,Grundlagen der Informatik`` (UE, 716.231)}%
  \newcommand{\courseics}{,,Einführung in das Studium der Informatik`` (710.104, Informatikstudium)}%
  \newcommand{\coursesee}{,,Einführung in Softwareentwicklung-Wirtschaft`` (710.010, SEW-Studium)}%
  \newcommand{\courseisp}{,,Einführung in die strukturierte Programmierung`` (706.001)}%
  \newcommand{\coursedm}{,,Diskrete Mathematik`` (503.007)}%
  \newcommand{\coursellp}{,,Logik und Logische Programmierung`` (716.051, Informatikstudium)}%
  \newcommand{\coursedsa}{,,Datenstrukturen und Algorithmen`` (708.031)}%
  \newcommand{\courseiis}{,,Einführung in die Informationssicherheit`` (705.026)}%
  \newcommand{\courseswp}{,,Softwareparadigmen`` (716.060)}%
  \newcommand{\coursetcs}{,,Theoretische Informatik 1`` (708.243, Informatikstudium)}%
  \newcommand{\courselc}{,,Logik und Berechenbarkeit`` (705.033, Informatikstudium)}%
  \newcommand{\coursedaa}{,,Entwurf und Analyse von Algorithmen`` (716.033)}%
\case{\equals{\languagename}{english}}%
%\case{{english}}%
  \newcommand{\studycs}{Computer Science [F 033 521]}%
  \newcommand{\studysdbm}{Software Development and Business Management [F 033 524]}%
  \newcommand{\courselfocs}{``Foundations of Computer Science'' (L, 716.232)}%
  \newcommand{\coursefocs}{``Foundations of Computer Science'' (P, 716.231)}%
  \newcommand{\courseics}{``Introduction to Computer Science'' (710.104, CS studies)}%
  \newcommand{\coursesee}{``Software Engineering and Economy'' (710.010, SDBM studies)}%
  \newcommand{\courseisp}{``Introduction to Structured Programming'' (706.001)}%
  \newcommand{\coursedm}{``Discrete Mathematics For Telematics'' (503.007)}%
  \newcommand{\coursellp}{``Logic and Logic Programming'' (716.051, CS studies)}%
  \newcommand{\coursedsa}{``Data Structures and Algorithms'' (708.031)}%
  \newcommand{\courseiis}{``Introduction to Information Security'' (705.026)}%
  \newcommand{\courseswp}{``Software paradigms'' (716.060)}%
  \newcommand{\coursetcs}{``Theoretical Computer Science I'' (708.243, CS studies)}%
  \newcommand{\courselc}{``Logic and Computability'' (705.033, CS studies)}%
  \newcommand{\coursedaa}{``Design and Analysis of Algorithms'' (716.033)}%
\otherwise
  \errmessage{Unknown valid language configured.}
\endswitch
%
\newcommand{\set}[1]{\left\{#1\right\}}
\newcommand{\spacewrap}[1]{\quad#1\quad}
\newcommand{\lequal}{\leftrightarrow}
\newcommand{\binval}[1]{\texttt{#1}} % binary value
\newcommand{\card}[1]{\hspace{3pt}\left| #1\right|\hspace{3pt}}
\newcommand{\On}[1]{\mathcal{O}\!\left(#1\right)}
\newcommand{\gramm}[1]{,,\texttt{#1}``}
\newcommand{\mathspace}{\hspace{20pt}}
\newcommand{\xmark}{\ding{53}}
\newcommand{\born}[1]{* #1}
\newcommand{\life}[2]{* #1 $\dagger$ #2}
\newcommand{\cNP}{$\mathcal{N\!P}$}
\newcommand{\cP}{$\mathcal{P}$}
\newcommand{\cNPC}{$\mathcal{N\!PC}$}
%
\makeindex
