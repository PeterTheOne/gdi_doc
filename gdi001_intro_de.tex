\chapter{Grundlagen der Informatik}
%
Informatik wird als Kombination aus zwei Wörtern beschrieben. Der
\emph{Information} und der \emph{Automatik}. Der Wortstamm wurde in
Sprachen wie Holländisch (informati-ka), Italienisch (informatica),
Französisch (informatique) und Polnisch (informa-tyka)
etabliert~\cite[p. 21]{Balzert04}.
Im Englischen wird Informatik
jedoch hauptsächlich mit dem Term \emph{computer science} bezeichnet,
welcher einen Fokus auf das Verb \emph{to compute} (berechnen) und
die Wissenschaft als solches legt. Wir möchten kurz erörtern in
welcher Beziehung Informatik zur Automatik, Berechnung, Wissenschaft
und Mathematik steht.

Informatik ordnet sich ähnlich der Mathematik als Ideal- oder
Formalwissenschaft ein. Bei dieser Form von Wissenschaft werden nicht
wie bei Naturwissenschaften reale Vorgänge gemessen, analysiert und das
zugrunde liegende Modell zur Erklärung des Vorgangs geschaffen. Es werden
Anforderungen an die Informatik herangetragen, wobei die Modelle und
Konzepte geschaffen werden, um diesen Anforderungen gerecht zu werden.
Informatiker benötigen Fähigkeiten aus den Bereichen des systematischen
Denkens, Modellieren, analytisches Lösen von Problemen und natürlich
Kreativität, um diese Konzepte entwickeln zu können.

Als Grundbaustein der Informatik dient Information. Information
wird als Summe von Daten (Sequenzen aus einem Alphabet) und Semantik
(Bedeutung der Sequenzen durch seine Relationen) betrachtet, welche sie
mit Maschinen automatisiert analysiert, bearbeitet und darstellt, wobei
sie wieder neue Information schafft. Die Automatik bzw. algorithmische
Erfassung der notwendigen Vorgänge ermöglicht es uns die entwickelten
Konzepte wiederzuverwenden und abstraktere Konzepte zu entwickeln.

Historisch betrachtet ist die Informatik eine junge Wissenschaft. Viele
eigenständige Wissenschaften haben sich erst in den letzten Jahrhunderten
etabliert. Der Einzug der Informatik in den Alltag datiert sich---je nach
Auslegung---keine 100 Jahre zurück. Die Informatik nutzt dabei als
Grundlage viele Konzepte der Mathematik. Die zahlentheoretischen
Zusammenhänge, das Transformieren von Zahlen mithilfe schrittweiser
Anleitungen und das Berechnen eines Gesamtsystems basierend auf ein paar
Grunddaten sind Konzepte, welche von der Mathematik übernommen wurden
und daher finden sich unter den frühen Informatikern
einige bekannte Mathematiker.

\index{Axiom}
Was versteht man jetzt als \emph{Grundlagen} der Informatik? Wie jede
Wissenschaft besitzt auch die Informatik Grundannahmen über die Welt,
welche verwendet werden. Diese Grundannahmen werden als \emph{Axiome} bezeichnet.
Sie erlauben es komplexere Zusammenhänge abzuleiten. Als Beispiel nimmt
die Mathematik (bzw. Arithmetik) etwa das Axiom \verb!1 + 1 = 2! an.
Die Informatik setzt etwa voraus, dass Zahlen in einer sequentiellen
Datenstruktur persistent gespeichert werden können. Erst dadurch wird die
Informatik praxisrelevant und erlaubt uns Berechnungen zu verwirklichen.
Es sind die Axiome, die wir in diesem Dokument betrachten wollen.

\index{4 Hauptgebiete der Informatik}
Die Informatik wird gerne in 4 Bereiche gespaltet:
\begin{itemize}
 \item Die \emph{theoretische Informatik} befasst sich mit
        den theoretischen Grundlagen und Grenzen der Berechenbarkeit,
        versucht Probleme zu klassifizieren und besitzt Schnittstellen
        zur Logik, Linguistik und Algorithmentheorie
        (Komplexitätstheorie, Automatentheorie, Formale Grammatiken,~\dots).
 \item Die \emph{technische Informatik} versucht
        reale Implementierungen zur Verfügung zu stellen, um Berechnungen
        zu ermöglichen und konzipiert die dafür notwendigen Maschinen.
        Sie besitzt insbesondere Schnittstellen zur Elektrotechnik bzw.
        Elektronik. Teilbereiche der technischen Informatik sind
        Prozessorarchitektur, Netzwerktechnik, Signalverarbeitung und
        hardwarenahe Systeme.
 \item Die \emph{angewandte Informatik} versucht all jene Tätigkeiten
        zu automatisieren und zu verbessern, die ein Informatiker
        selbst im Alltag benötigt. Große Felder dieses Bereiches sind
        etwa Programmiersprachen, Betriebsysteme oder Datenbanken. Da
        sie die Informatik selbst bedient, besitzt sie keine
        Schnittstellen zu anderen Wissenschaften; verwendet jedoch
        Elemente der Logik, Mengenlehre und Architektur von Maschinen.
  \item Die \emph{praktische Informatik} versucht die Konzepte der
        Informatik in anderen Wissenschaften (etwa Wirtschaft, Medizin,
        Biologie) anzuwenden. Sie nimmt explizit die resultierenden
        Werkzeuge aus den anderen Gebieten der Informatik an, um sie auf
        die Modelle der anderen Wissenschaften anzupassen.
\end{itemize}
